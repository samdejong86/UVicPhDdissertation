\newpage
\TOCadd{Abstract}

\noindent \textbf{Supervisory Committee}
\tpbreak
\panel

\begin{center}
\textbf{ABSTRACT}
\end{center}

The Belle II detector is designed to collect data from the high luminosity electron-positron (e$^+$e$^-$) collisions of the SuperKEKB collider. It will explore the flavour sector of particle physics through precision measurements. The backgrounds of particles induced by the electron and positron beams will be much higher than in any previous \epem collider. It is important that these backgrounds be well understood in order to ensure appropriate measures are taken to protect the Belle II detector and minimize the impact of the backgrounds. In February 2016 electron and positron beams were circulated through the two 3 km vacuum pipe rings without being brought into collision during `Phase I' of SuperKEKB commissioning. Beam backgrounds were measured using Belle II's commissioning detector, BEAST II. BEAST II is composed of several small subdetectors, including helium-3 thermal neutron detectors. The BEAST II thermal neutron detector system and results from its Phase I running are presented in this dissertation. The Phase I experiment studies beam-gas interactions, where beam particles collide with residual gas atoms in the beampipes, and beam-beam interactions, where beam particles interact with each other. Simulations of these two types of backgrounds were performed using the Strategic Accelerator Design (SAD) and GEometry And Tracking (GEANT4) software packages. A method to account for the composition of the gas in the beampipes was developed in order to correctly analyse the beam-gas component of the background. It was also determined that the thermal neutron rates in the data on the positron beam were 2.18$^{+0.44}_{-0.42}$ times higher than the simulation of beam-gas interactions and 2.15$^{+0.34}_{-0.33}$ times higher for beam-beam interactions. The data on the electron beam were 1.32$^{+0.56}_{-0.36}$ times higher for beam-gas interactions and 1.91$^{+0.54}_{-0.48}$ time higher for beam-beam interactions. The impact of these studies on Belle II is discussed.





