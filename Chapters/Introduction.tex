\chapter{Introduction}
\label{chap:Intro}

The Belle~II detector is designed to collect data from the high luminosity SuperKEKB electron-positron (e$^+$e$^-$) collider. It will explore the flavour sector of particle physics through precision measurements and will reach particle interaction rates never before achieved in an \epem collider experiment. As such, backgrounds generated from the beam will also increase dramatically. 


Beam particles can be lost from the beam through three mechanisms: \epem interactions, interactions of the beam with residual gas in the beampipe, and interactions of beam particles with other particles in the same bunch or group of particles (the Touschek effect). This dissertation focuses on the latter two mechanisms. Particles lost through the beam-gas collisions and the Touschek effect can interact with the beampipe, producing showers of particles including neutrons. Neutrons produced can be slowed down by interaction with materials around the beampipe. These thermal neutrons can cause degradation of Belle~II's performance and even cause damage to the detector. Simulations of these backgrounds and the neutron flux they produce have been performed, but it is important to measure the backgrounds and determine corrections to the simulation and uncertainties on these corrections.




In order to measure the beam backgrounds before the Belle~II detector is installed, an apparatus called BEAST~II is placed around the point where the electrons and positrons will collide. BEAST~II will run for three phases. Phase~I, a skeletal collection of small subdetectors, ran February -- June 2016. There were no collisions between the electrons and positrons during this phase.  Phase~II will be composed of most of the Belle~II detector, without the vertex detectors, and will start running in late 2017. Collisions of electrons and positrons will begin at this point. The vertex detectors will be installed in Phase~III, and the Belle~II experiment will begin in full. The purpose of BEAST~II is to answer these questions: How accurate are the simulations of beam-gas and Touschek backgrounds? Do upgrades to Belle~II's subdetectors need to be considered? Is more shielding required?




BEAST~II is composed of several subdetectors which measure various types of radiation. One of these detectors is a set of four thermal neutron detectors. These detectors are stainless steel tubes which contain pressurized $^3$He. When a neutron collides with a $^3$He nucleus, the nucleus splits into a proton and a tritium ($^3$H) nucleus. These produce ionization in the tube, which is measured with a sense wire at the centre.





The components of the Belle~II detector are described in Chapter 2. Phase~I of BEAST~II is described in Chapter 3, as well as comments about Phase~II. The \he thermal neutron detector system is described in Chapter 4, along with details about the calibration, location in Phase~I, and magnetic field tests. The sources of beam backgrounds expected in Phase~I are discussed in Chapter 5. The experiments performed in Phase~I to measure these backgrounds are described in Chapter 6. An explanation of how the simulation of Phase~I was performed and weighted is given in Chapter 7. The techniques used to analyze the data recorded in Phase~I are demonstrated in Chapter 8. The consequences of the studies performed in Phase~I of BEAST~II for full Belle~II running are discussed in Chapter 9, followed by closing remarks in Chapter 10.





