\chapter{Conclusion}
\label{chap:Concl}



The studies described here show that the simulated flux of neutrons from beam-gas and Touschek interactions are underestimating the actual neutron flux by 2.18$^{+0.44}_{-0.42}$ and 2.15$^{+0.34}_{-0.33}$ for beam-gas and Touschek on the LER respectively, and 1.32$^{+0.56}_{-0.36}$ and 1.91$^{+0.54}_{-0.48}$ for beam-gas and Touschek on the HER respectively. The ramification of this is that Belle~II will experience a 20\% higher than expected total neutron flux for these sources. The detectors will have a neutron flux at a higher rate than was initially expected and some detectors will have a neutron flux above their tolerance.

When Phase~II of BEAST~II runs, there will be collisions, which will allow the radiative Bhabha component of the backgrounds to be measured. The \he tubes will be present at this time and thus the neutron rate due to these Bhabhas can be measured for comparison with the simulation.

During Belle~II's physics running, the Belle~II detector will trigger data acquisition at random times when the beams are being circulated. This will allow for real time measurement of beam backgrounds. It is likely that one or more of the \he tubes will be used to measure the neutron flux during the full Belle~II experiment, as a monitor on the neutron flux from the beam backgrounds.






